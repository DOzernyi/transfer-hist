\documentclass{article}
\usepackage[utf8]{inputenc}
\newcounter{glossnum}
\newcommand{\numgloss}{\refstepcounter{glossnum}\alph{glossnum}.\space}
\usepackage[dvipsnames]{xcolor}
\usepackage{tikz}
\usetikzlibrary{positioning}
\renewcommand{\theglossnum}{\the\excnt\alph{glossnum}}
\usepackage[ukrainian, english]{babel}
\captionsenglish
\usepackage{multicol}
\usepackage[linguistics]{forest}
\forestset{qtree edges/.style={for tree=
    {parent anchor=south, child anchor=north}}}
\usepackage{natbib}
\usepackage[colorlinks=true, citecolor=blue, linkcolor=blue, urlcolor=blue]{hyperref}
\usepackage{qtree}
\usepackage{expex}
\usepackage{dirtytalk}
\usepackage{mathtools}
\newcommand\Set[2]{\{\,#1\mid#2\,\}}
\newcommand\SET[2]{\Set{#1}{\text{#2}}}
\usepackage[
    protrusion=true,
    activate={true,nocompatibility},
    final,
    tracking=true,
    kerning=true,
    spacing=true,
    factor=1100]{microtype}
\SetTracking{encoding={*}, shape=sc}{40}
\usepackage{amsmath}
\usepackage{amsfonts}
\usepackage{qtree}
\usepackage{gb4e}
\noautomath
\renewcommand{\UrlFont}{\ttfamily\small}

\let\OLDthebibliography\thebibliography
\renewcommand\thebibliography[1]{
  \OLDthebibliography{#1}
  \setlength{\parskip}{0pt}
  \setlength{\itemsep}{0pt plus 0.3ex}
}


\title{The rise and fall of linguistic transfer\\
{\fontfamily{qcr}\selectfont Draft}\footnote{Cite as: Ozernyi, Daniil M. (2022). \textit{The Rise and Fall of Linguistic Transfer}. Manuscript, Northwestern University. \href{https://doi.orgxxx}{DOI:}. \\ Parts of this paper were excerpted for a proceedings paper for the $96^{th}$ Annual Meeting of the Linguistic Society of America \textit{Some remarks on history of transfer in language studies} (available at \href{https://dozernyi.com/work}{dozernyi.com/work} or $PLSA$ archives website). I hope this paper makes it to print at some point, but it's available as a manuscript meanwhile.}}
\author{Daniil M. Ozernyi\\ Department of Linguistics, Northwestern University}
\date{February 2022}

\begin{document}

\maketitle

\section{Introduction}
The notion of transfer\footnote{For both parsimony and convenience reasons, I will use ``transfer'' to mean ``the notion of transfer'' and ``the process of transfer'', the two being roughly synonymous for the purposes here. Where I will diverge from this convention, I will specify the intended meaning.} is crucial to many a modern subfield of linguistics. This is primarily due to the fact that one of the main objectives of language acquisition studies beyond \(L_1\) is to establish the role of the previous languages in acquisition of the subsequent ones \citep[cf.][]{epstein_second_1996, cenoz_beyond_1998, cenoz_cross-linguistic_2001, white_universal_2012, rothman_third_2019}. Trivially, the notion of transfer is cornerstone in such an inquiry, or at least so it has been since studies on second language appeared as a branch of psychology and education studies\footnote{Note, however, that studies on second language generally span way further back, e.g. ``Augustine's famous reflections on learning $L_1$ Latin and $L_2$ Greek'' \citep[][p. 743]{thomas_full_1996}. See a more extensive overview in \cite{thomas_medieval_1995}.} \citep[since][]{lado_linguistics_1957}. Some researchers even refer to these studies as ``transfer studies'' \citep[e.g.,][]{gass_second_1988, ringbom_importance_2006, puig-mayenco_systematic_2020, puig-mayenco_low_2020}. Since it hardly lends itself to debate that transfer and the adjacent notions (e.g. cross-linguistic influence (CLI), interference, etc.) are at the very core of the field, one might expect them to be well-defined and their definitions to have been long agreed upon. This, however, does not appear to be the case. 

Even a brief survey of the recent papers in the field of third language acquisition shows that there is still a lot of debate, if not confusion, surrounding transfer. One such example is Rothman \textit{et al}.’s distinction between transfer and CLI \citep{rothman_third_2019}, and Westergaard’s defeasance of such a distinction \citep[][p. 104]{westergaard_plausibility_2021}; another brief critique of transfer vs. CLI can be found in a review \citep{ozernyi_jason_2021} of Rothman \textit{et al}. (op. cit.). The chief argument of Ozernyi in his review is the transfer is not well-defined, hence it cannot be distinguished in any precise or meaningful way from cross-linguistic influence -- or from anything (e.g., an apple\footnote{How do you tell transfer and apple apart? You define apple, you define transfer, and you prove that apple \(\neq\) transfer. If you have not defined either apple or transfer, inequality cannot hold because transfer might have the same properties as the apple does.}) for that matter -- \textit{a priori}, since the distinctions that are being made need to be metaphysically precise. Evidently, there is debate as to what transfer is and what it entails. It appears as though transfer came to be an umbrella term for any influence of any trace of \(L_{1,2,3...n}\) on some \(L_{n+1}\). 

In any case, negligence with terminology has far-reaching consequences\footnote{For one of the more famous examples of exposing such consequences, see \cite{chomsky_review_1959}, particularly the first couple of sections.}. It is daunting indeed that the notion which effectively establishes the \textit{modus operandi} of an entire field is ill-defined. Such neglect renders the research hard or impossible to interpret and obscures the exchange of ideas. Are we to assume that sociolinguistic/sociolcultural transfer \citep[e.g.,][]{fouser_too_2001} and transfer or gender, transfer of lexicon, transfer of phonology are guided by one single operation -- transfer? This ambiguity is the main reason why it is vitally important to take a more thorough, in-depth look at the history of transfer: in hope that its history might help us in a quest to understand transfer better and perhaps draw much finer boundaries between transfer and adjacent terms. In addition, the hope is that this essay will lend itself to didactical purposes for those working or preparing to work in \(L_n\) acquisition; there is little doubt that any researcher should be constantly reminding themselves of imperative wariness and prudence while handling terminology.

The structure of the essay is as follows:

\begin{itemize}
    \item in Section 1, I attempt to trace back the source of the title term, and that quest takes us back to \cite{priestley_hartleys_1790} and \cite{james_principles_1890-1}; I then try to formalize what Priestley and James might have meant by transfer and explain why formalization is important;
    \item in Section 2, I look at work in psychology which followed James (op. cit.) up to \cite{lado_relation_1949} and give an outline of the rise of transfer in academic literature. The focus is on psychological studies since language learning research had not branched out of psychology until the late 1940s; I also comment on how transfer changed and expanded through that period;
    \item in Section 3, I contemplate the ingress of transfer to linguistic and language studies, which happened mainly within the papers of Robert Lado \citep[including but not limited to][]{lado_relation_1949, lado_survey_1950, lado_testing_1951, lado_comparison_1956, lado_linguistics_1957, lado_sentence_1957}. I analyze Lado and his contemporaries’ interpretation and use of transfer, and show how it changed since its emergence. I also point out that in almost no studies of that period can one manage to find any actual definition of transfer;
    \item in Section 4, I observe the evolution of transfer within linguistics after Lado’s work to the 1990s and modern day. As stated elsewhere, I focus mostly on generative inquiry, and do not follow the evolution of understanding of transfer in conceptually differing approaches such as behaviorism or respective areas language teaching \citep[along the lines of][]{tomasello_feedback_1989}; 
    \item in Section 5, reflections on the findings relating them to the contemporary studies as well as some proposals are offered;
    \item lastly, Section 6 is a brief conclusion.
\end{itemize}

\section{The emergence and initial development of transfer: before the 1900s}

To better understand the nature of the notion of transfer as it is used in language studies, one needs to find the last instance where it was used to mean roughly the same process it is used to signify today. This, naturally, will be outside of the scope of cognitive science and linguistics in the modern sense. In addition, I intentionally will not define transfer here, at least not until a further section, which leaves us operating in quite vague terms but beneficially removes the possible bias towards one definition or another.

One of the first references to transfer cited in later relevant literature \citep[i.e., psychology of the 1890s-1920s;][]{james_principles_1890-1} appears to be in Priestley when he talks about the nature of judgement as feeling: ``[Judgement is] transferring the idea of truth by association from one proposition to another that resembles it'' \citep[][p. 30]{priestley_hartleys_1790}. Priestley does not himself give a definition, apparently taking it to be self-evident. This is the affliction of many future papers which, implicitly amending or adjusting the term, used it without defining.

There are two crucial elements of transfer which we can infer from the Priestley’s use of it:

\pex Properties of Priestley's transfer
\a transfer presumes assigning a property which is relevant to one item A to another item B, and
\a it is imperative that A and B are associated, i.e. A ``resembles'' B.
\xe

One century later, Priestley was quoted by William James in his pioneering \textit{Principles of Psychology} (\citeyear{james_principles_1890-1})\footnote{I absolutely do not purport to claim that transfer was not used in-between. Priestley was chosen, however, because only four citations separate him from Lado (\citeyear{lado_testing_1951}), and only five citations separate Priestley from Flynn (\citeyear{flynn_microvariation_2021}).}. This was perhaps one of the entrance points for transfer to appear in psychology (in contradistinction to Priestley’s philosophical work). James does not give us a definition either. He starts out with Priestley’s words almost \textit{sic erat scriptum}, writing about ``transfer of feeling from one object to another, associated by contiguity or similarity with the first'' \citep[][p. 330]{james_principles_1890-1}. However, it is obvious that later on James expands transfer as he talks about ``transfer of relations [...] within a homogenous series'' \citep[][p. 660]{james_principles_1890}).

Now, let us turn to formalization so that the properties of transfer from the quotes above can be stated formally, hence unambiguously. Formalization of the terms one uses is of great importance for establishing and advancing the clarity of terminological and methodological apparatus (see \cite{collins_chomsky_2021}; and briefly in \cite{zaccarella_neuroscience_2021} as well as references therein). Formal definition, insofar as it can be ascertained, leaves very little for space for ambiguity. It can serve especially well when scrutinizing the phylogeny, in our case that of the transfer. For Priestly, transfer seems to be similar or identical to copying, i.e. an action which takes some set of properties \textit{A} 
\begin{align*}
A  &\coloneqq \Set{p_1...p_n}{\textit{p} \textrm{ is an abstract property}}
\end{align*}
from one item X to which the set A is assigned, copies them, and then assigns them to a distinct item Y. In Priestley’s case, he seems to imply that an individual takes a set of properties of one idea X (for example, a singlet connoting that the idea is true: i.e, \(A = \{T\}\)) and transfers it to a different idea Y so that the latter is also assigned \{T\}. There is no indication that either A or \{T\} (or B, modulo the addition of a property) are changed in any way in the process. Thus, we have:

\pex Transfer (interpretation of Priestley 1790)
\a Take \textit{X} to be some abstract item, and set \textit{A} as defined above to contain the properties of \textit{X}, and further take Y to be an item distinct from X. 
\a Then \textit{transfer} is the process by which at least one element of set \textit{A} of X is copied and assigned to Y.
\xe

This formalization raises some questions, though. For example, if we have an item A which has a set of properties $\{p_1...p_n\}$ where $n>2$, is it imperative that the whole set transfers or can parts of it be selected for transfer? 

James’ use differs from that of Priestley. The readily available interpretation of James is just extending Priestley: the transfer becomes a process of copying some relation \textit{R} (could be taken to be roughly a function) between two items \{X, Y\}, i.e. \(R(X, Y)\) (analogous to copying a set of shared properties \(A_X \cap A_Y\) provided it's nonempty)\footnote{The notation here and further is somewhat unconventional as it is related to merge (both in computer science and in syntax) and related notions, but serves the expository purpose well and I believe is intuitive and unambiguous in the context.} and assigning it to two different items \{X', Y'\} obtaining thus \(R(X', Y')\). This, however, faces some problems. First is that \{X, Y\} and \{X', Y'\} need to be distinct sets in order for them to be considered a minimal homogeneous series. In other words, if we take two items (a minimal series) \{X, Y\} and transfer their property which they share to \textit{n} other series like them $\{X_n, Y_n\}$, we will still be getting copies of the same series and not one continuous, expanding homogeneous series: $\{X_1, Y_1\}, \{X_2, Y_2\}, \{X_3, Y_3\}$, $\{X_n, Y_n\}$. This does not seem to be right because James talks about transfer ``within'' one series.

What James’ variant might instead imply is that when some property $\{p_i\}$ is transferred from item $X_1$ to item $X_2$, then from $X_2$ to $X_n$, items $X_1, X_2, ...X_n$ become a homogeneous series $\{X_1, X_2, X_3... X_n\}$. There are many ways to think of this operation and its properties, one of the more straightforward is some binary set formation operation, call it $\omega$. Thus, we have:


\pex Transfer (interpretation of James)
\a Take \textit{X} to be some item, and $\{p_1...p_n\}$ to be some properties which \textit{X} has (i.e., $\{p_1^X...p_n^X \in X\}$), and \textit{Y} to be an item distinct from \textit{X} such that $\{p_1^Y...p_n^Y \in Y\}$.
\a Then transfer can be stated as\\ $\omega$($\{p^Y_1...p^Y_n\}$, $\{p^X_1...p^X_n\}$) $\rightarrow$ $\{\{p^Y_1...p^Y_n\}, \{p^X_1...p^X_n\}\}$.
\xe

Of course, there are many issues with the ``formalizations'' above -- in terms of set-theoretic notions employed, in terms of relations to the original statements in the papers of Priestley and James, etc. The main point, however, is that the definitions or context which we considered here are readily formalizable in some shape which makes it possible to understand them better. Foreshadowing, as soon as transfer becomes something of an applied term primarily in experimental psychology, it will lose its precision and clarity. 

\section{Transfer in psychology: the 1900s to the 1940s}

Subsequently, James’ transfer, vague enough to be interpreted in many ways (recollect that we never defined what item \textit{X} or item \textit{Y} were -- they could be related to psychology as well as to language, etc.), entered and took its place in psychology. Psychology, in turn, swelled and prospered after the 1890s, and many studies were conducted, including those on transfer. Incidentally, while James did not use the term transfer omnipresently or even consistently, it became very much central to many psychological inquiries in the 1910s and even beyond. It is enough to look at the titles (e.g., Winch’s \textit{Transfer on Improvement in Memory…} (\citeyear{winch_transfer_1908}) or Wallin’s \textit{Two Neglected Instances in the Transfer of Training} (\citeyear{wallin_two_1910})) to see how reliant on this notion the studies were. Transfer also made recurrent appearances in psychology textbooks \citep[][e.g.,]{wheeler_science_1929}. 

This essay will intentionally continue looking at the psychological studies up to Lado (\citeyear{lado_relation_1949}), the first time transfer made its appearance in second language learning. I see it as important to have a full grasp of how wide-spread and pervasive the usage of transfer had become by the late 1940s. This ubiquitousness, naturally, lays the foundation for future misunderstandings in the use of the term. Where possible, I will point out studies relating to language or language learning. 

Some notable studies in the 1900s and early 1910s looked into transfer in the domains of memory \citep{winch_transfer_1908, dearborn_experiments_1910, peterson_note_1912, fracker_transference_1907}, sensual, perceptual, or motor function (for a pioneering study, see \cite{woodworth_influence_1901}; but also \cite{scholkov_relation_1908}; Wallin op. cit.), and some are rather more peculiar (e.g. \textit{Experiment on Transfer of Ideals of Neatness} in \cite{bagley_experiment_1905}). However, no definition was given in those studies: transfer maintained its subdoxastic nature. Incidentally, the 1910s is the first time when some suspicion towards transfer is voiced. Thorndike and Woodworth refer to a ``\textit{mysterious} transfer of practice [...], an unanalyzable property of mental functions'' \citep[][p. 256, emphasis added]{woodworth_influence_1901}. Their phrasing signals a lack of contemporary understanding of the mechanisms of transfer, and \textit{ergo}, its origin. 

In the mid-1910s the previous studies were summarized in the seminal monograph by Rugg (\citeyear{rugg_experimental_1916}). He, however, did not differ much from his predecessors in not defining the transfer. Rugg’s work indicates further expansion of transfer: he is referring back to James, substituting James’ ``restitution of knowledge'' by his own ``transfer''.  The work following Rugg brought even more adjacent terms: ``transfer[ring] … capacity'' \citep[][p. 406]{downey_automatic_1915} and ``transfer effect'' (407); all of which were cognates. 

Later studies generally proceeded with the still undefined notion, including those\footnote{Many of the references which follow come either from McGeoch (\citeyear{mcgeoch_psychology_1942}) or from Rugg (\citeyear{rugg_experimental_1916}). See McGeoch (\citeyear{mcgeoch_influence_1930}) or Bills (\citeyear{bills_general_1934}) for more comprehensive bibliographies which include studies from the 1920s and the 1930s.} on learning curves and memory \citep{martin_improvement_1929, dallenbach_effect_1914}; motor or psychophysical functions \citep{fernberger_effects_1916}; also see \cite{leuba_studies_1905} on ``writing English prose in German script'', on nonsense syllables \cite{melton_influence_1940}\footnote{Upon examination, I do not find that these studies hold enough value to be described in detail or to be re-analyzed within modern approaches to language learning. Despite involving language, the studies are blatantly behavioristic which is natural per the dominant approach of the time.}), rational learning \citep{ruger_psychology_1910, mcgeoch_influence_1930}, more on positive transfer \citep{sleight_memory_1911, reed_repetition_1917, mudge_transfer_1938}; one negative transfer \citep{archer_transfer_1928}; on motor function \citep{bills_general_1934} and references therein). Notably, in the 1930s even more studies on transfer in the context of language appeared \citep[e.g.,][]{johnson_language_1933}.

The first elaborate and clear definition of transfer we get is in the 1940s from McGeoch\footnote{Those interested in a more comprehensive understanding of McGeoch’s elaboration would be well-advised to visit Chapter 10 of \textit{The Psychology of Human Learning, An Introduction} which is entirely devoted to ``transfer of training''.}:

\pex Definition of transfer and adjacent notions in \cite{mcgeoch_psychology_1942}
\a The influence of prior learning (retained until the present) upon the learning of, or response to, new material has traditionally\footnote{Note how McGeoch appeals to tradition rather than citing any of the studies, albeit in the rest of the book he is admirably punctilious about the terminology employed and his definitions (for an example of this see his fn. 12 and elsewhere).} been called \textbf{transfer of training}. 
\a It appears in experimental measurements as a \textbf{transfer effect}, which means the influence of a specified amount of practice or degree of learning in one activity upon the rate of learning of another activity or upon response to another situation\footnote{This would go on to be at the foundation of contrastive analysis of Lado and Fries. We will talk about their usage of transfer in the next section.}. 
\a Transfer effects may be (a) \textbf{positive}, when training in one activity facilitates the acquisition of a second activity, (b) \textbf{negative}, when the training in one inhibits or retards the learning of another, and (c) \textbf{zero or indeterminate}, when training in one has no observed influence on the acquisition of a second. \citep[from][p. 394, emphasis added]{mcgeoch_psychology_1942} 
\xe

McGeoch goes to great lengths to discuss various kinds of transfer and effects transfer might have. There is little doubt that this work influenced subsequent induction of transfer to linguistics\footnote{As evidenced, among others, by the citation of McGeoch (\citeyear{mcgeoch_psychology_1942}:55-56) in Lado (\citeyear{lado_relation_1949}).}. Again, a couple of tokens seem to indicate that transfer expanded: McGeoch refers to transfer in Sobel’s work whereas the latter actually used ``transference'' \citep[cf.][p. 386]{sobel_study_1939}. Transfer or its variants were also used now to explain learning curves (see McGeoch \citeyear{mcgeoch_psychology_1942}:45f; Pechstein \citeyear{pechstein_experimental_1939}:41, and references therein) which is not observed (or not as often) in the studies from the 1910s and the 1920s. 

At this point, it is fairly hard to formalize transfer and its role as we did at the beginning with Priestley and James. There, however, are some significant changes to the definition. From copying a property or copying a relation we went to much more vague ``influence''. That is, going back to formalizing, there is no copying anymore: there is some ``influence'' of item \textit{X} upon item \textit{B}. This change is rather important. While the studies seemingly became more precise and investigated what they deemed to be ``the transfer effect'', the definition did not narrow down. Conversely, while retaining its general properties from James (\citeyear{james_principles_1890-1}), the definition somewhat expanded (see 4-7)\footnote{Trivially, there were some changes which we are considering explicitly because they are field-specific, and are not in and of themselves formalizable (e.g., ``specified amount of practice''). We are primarily interested in what McGeoch’s definition would look like if formalized and stripped of time-specific and field-specific elements.}:

\pex Expansion of transfer in James vs. McGeoch
\a If we look at the pre-1900s transfer, the property ``new'' (McGeoch’s ``new material'') was never mentioned. That is, in the old definition, transfer was meant to occur within some system or workspace, and the definition did not necessitate introducing something (new) into that system.
\a Another expansion is of the definition of negative, positive, and zero\footnote{It is not clear why it was necessary to introduce a purely stipulative notion of zero transfer instead of saying transfer was absent.} transfers. That is, any ``acquisition'' process was taken to be accompanied by some kind transfer.
\a Since in McGeoch transfer was taken to be an influence as opposed to a function (in our rather liberal interpretation of James), it is hard to tell what the interaction between item \textit{X} (which is now previous knowledge) and item \textit{Y} (which now is being acquired) is. In other words, it remains unclear whether \textit{X} (previous) is altered after transfer or copying of its properties to \textit{Y} (new/target).  
\a While James mentions homogenous series and transfer within such series, McGeoch does not. Similarly, note how McGeoch omits any mention of ``congruity or similarity'' between two items engaged in transfer. Transfer now does not bear any trance of the two elements from \cite{priestley_hartleys_1790} which we distinguished earlier. 
\xe 

There, however, is some useful clarification in McGeoch’s definition. Namely, his point that in order to be transferred, properties should be ``retained upon the present''. To the best of our ability:

\pex Transfer (interpretation of McGeoch 1942)
\a Take \textit{X} to be a system with some items $X_1...X_n$, each of which has sets of properties $\{p_1...p_n\}$, 
\a then transfer is the influence (or absence thereof) which the properties $\{p_1...p_n\}$ have on introducing to the system the item or the items $Y_1...Yn$ with their respective sets of properties.
\xe

Naturally, nothing is quite clear in this definition: what is a system? what are the items? what is the influence? If for our interpretation of James, copying is easy to define and item can be defined depending on the domain of transfer -- ``influence'' of McGeoch is hopelessly vague, hence unsuitable for formalization.

Were we to pose the question in a different way, namely, what stayed impervious between our interpretation of James’ and McGeoch’s definition, it is fairly hard to answer. Items \textit{X} and \textit{Y} are still present in one way or the other. We stipulated sets of properties, but there is little in McGeoch to suggest that he implied it that way. Other than that, nothing seems to be common, i.e. the definition of transfer changed drastically in formal terms -- and yet nobody noted that in the literature; the transfer was taken to be too trivial to define, retaining its subdochasticity. The analysis above shows that our alarmedness towards transfer is neither delusional misidentification syndrome \citep[cf.][]{christodoulou_delusional_1991, feinberg_delusional_2005} nor a vain game in the dilemma of the Ship of Theseus (cf. Heraclitus' \textit{Cratylus}), but a legitimate concern about the notions which scaffold the science we do. 

\section{Transfer’s induction to linguistics and language science: 1949-1957}

Two figures which pioneered language learning and, specifically, what later came to be known as contrastive analysis (CA), were Charles C. Fries and Robert Lado. Early work of Fries did not concern foreign language learning: his papers were mostly on structure of English as a language and learning English a first language \citep[cf.][]{fries_periphrastic_1925, fries_meanings_1927, fries_rules_1927, fries_teaching_1927}. However, even in later works, he does not appear to use transfer. In his seminal textbook on teaching English as a foreign language, transfer is nowhere to be found \citep{fries_teaching_1945}.  Instead, what seems to be the very first instance of usage of transfer in a paper\footnote{I am careful to note ``in a paper'' because in 1948 in the \textit{Language Learning} journal (the first year of it being published), there was a mention of transfer in editorial: ``Errors induced by transfer from the student’s native language will not be influenced by a few incidental exposures without intent to learn'' \citep[][p. 3]{l_editorial_1948}. However, the author of the editorial is unknown -- it was signed R. L., and there wasn’t anyone with those initials except Robert Lado at that point in the journal. However, Robert Lado in 1948 was responsible for advertising and had little to do with editorial duties -- so, the nature of the editorial remains unclear, even though it is likely that it was, after all, written by Robert Lado.} on foreign language learning is Lado (\citeyear{lado_relation_1949}). In a footnote on this fragment:

\begin{quote}
    From a psychological point of view we note that the learner will acquire more rapidly those elements of the foreign language that operate on habits already established for the native language, less rapidly those elements that require the acquisition of new habits, and least rapidly those in which the new habits conflict with the linguistic habits already established by the native language \citep[][p. 109]{lado_relation_1949}
\end{quote}

he refers to McGeoch’s transfer (\citeyear{mcgeoch_psychology_1942}:55-59). Approximately at the same time, Fries and Pike in a paper on phonology, mention ``transfer from Spanish to English nasals`` \citep[][p. 37]{fries_coexistent_1949} and reference Marckwardt (\citeyear{marckwardt_phonemic_1946}), despite the fact that Marckwardt did not use ``transfer'' and used ``influence'' instead (111). Later on, Lado spearheaded the campaign on using transfer, it looks like, because a number of his works which we will take a more careful look at below make exceptionally wide use of the term, including but not limited to Lado (\citeyear{lado_survey_1950}, \citeyear{lado_testing_1951}, \citeyear{lado_comparison_1956}, \citeyear{lado_linguistics_1957}, \citeyear{lado_sentence_1957}). 

Lado seems to have introduced the term of wholesale transfer which is thriving today (albeit hopefully, yet arguably) in a different meaning \citep[cf.][]{schwartz_full_2021, westergaard_microvariation_2021}\footnote{Westergaard does not support the wholesale transfer models of \(L_n\) acquisition, merely makes wide use of the term \citep[][pp. 2, 12f, 15, etc.]{westergaard_microvariation_2021}}: ``a \textbf{wholesale} transfer of a reading technique into aural comprehension...'' \citep[][p. 53]{lado_testing_1951}, emphasis added). It is clear, however, that at this point, transfer is still not being used in modern, ``linguistic'' meaning, an example of which would be transfer of parameters or features (properties which can take different shape depending on the theoretical framework one chooses) from the previous language to the target language. Instead, what we see now in Lado's work is the introduction of ``psychological'' transfer-of-training to linguistics and language learning. The difference between the two will emerge later and will become increasingly pronounced by Zobl \cite{zobl_developmental_1980}. The talk of ``structures'' appears already in 1950: ``those \textbf{structures} in the foreign language that are not transferable from the native language are the ones we seek to discover by comparing the two languages in order to have the most effective testing materials'' \citep[][p. 19, emphasis added]{lado_testing_1951}. Here’s a useful nascent notion: that of transferability; it is yet another newly-introduced term to account for those structures which are ``not transferable''. Why, however, any given structure was thought to be ``not transferable'' is not clarified.

Later on in the book, we also get a more elaborate description of what ``wholesale'' transfer is:
\begin{quote}
    a speaker of one language tends to transfer the entire system of his language to the foreign language[...]. He tends to transfer his sound system, including the phonemes, the positional variants of the phonemes, and the restrictions on distribution. He tends to transfer his syllable patterns, his word patterns, and his intonation patterns, as well \citep[][p. 26]{lado_comparison_1956}).
\end{quote}

This view echoes more recent work (e.g., the ITH model \citep{leung_third_2007} as well as the initial work on the TPM model, see references above), and surpasses the initial work of Lado on phonology, augmenting it with ``word patterns''. Under this definition, transfer is not selective (``the entire system''). Oddly, this view contradicts Lado's earlier \citep{lado_testing_1951} mention of non-transferrable structure, i.e. those parts of ``system'' which do not transfer, which would make the ``wholesale transfer'' or the transfer of ``the entire system'' simply impossible. Such incoherences are prominent, not only in Lado’s work, but overall in transfer literature. In Lado, however, they are particularly pronounced. In addition to the the issue with transferability, for Lado, learners\footnote{At least, the students of the first-year college courses which he is talking about in the 1951 paper quotes immediately above.} are \textit{conscious} of transfer and unwilling to accept it: ``in spite of [them]self [the learner] will transfer those habits to the new dialect and styles [they are] trying to learn'' \citep[][p. 14]{lado_sentence_1957}. Lado also mentions ``intent'' in earlier work (see Lado op. cit.: fn. 13).

The last paper of Lado we’ll consider here is his seminal work on CA -- \textit{Linguistics across cultures} \citep{lado_linguistics_1957}. Notably, he references transfer at the very beginning, alluding to Fries (\citeyear{fries_teaching_1945}), despite Fries not using transfer in his book. Similarly, there is a reference to Dreher’s work on ``transfer of intonation'' despite Dreher not using transfer in his doctoral thesis \citep{dreher_comparison_1950}. But we have seen this pattern many times. Apart from re-stating his earlier theses on nature of phonetic and phonological transfer \citep[][p. 11]{lado_linguistics_1957}, Lado expands it to ``physically similar phonemes'' (12), transfer of morphology (58), even reading habits (94) and writing system\footnote{Lado, acknowledges, however, that ``we are less clear on how this transfer will affect our learning of a foreign language writing system`` \citep[][p. 97]{lado_linguistics_1957}} (97); he mentions positive/negative transfer as well (109). 

The definition of transfer in Lado’s work is nowhere to be found. What we find, however, is neither a purely ``psychological'' transfer (``transfer of training'', of habits\footnote{This kind of transfer was found in \cite{rugg_experimental_1916} and \cite{mcgeoch_psychology_1942}.}), nor a purely \textit{ante litteram} linguistic one  (\textit{sc.} transfer of structure, perhaps of mental representation, but definitely not of habits or conscious activity or (conscious) metalinguistic competence\footnote{It is important to note that metalinguistic competence is a notion that is not well-defined. It is taken to mean roughly ``conscious insight about language'' \citep[take, e.g.,][]{bardel_role_2007, bardel_l2_2017}. I take it to mean thinking process along the lines ``move out the auxialiary to form a question'', a rule which the learner was taught and which is not done unconsciously, but more like math. However, with practice and time, does the learner internalize the grammar and needs not those ``conscious rules'' or does the learner just get proficient at the rules so that timing shrinks significantly? The answers, and the precise line between acquisition and learning, and the role learning plays in acquisition, hypothesis space for the two processes -- all are  yet to be clarified.}). Instead, Lado seems to present a cludge of the above two: transferring reading and writing habits obviously are instances of ``psychological'' transfer; while gender, case, and other morphological features are obviously a much more subtler, unconscious, ``linguistic`` transfer\footnote{My use of unconscious ``linguistic`` vs. conscious ``psychological'' distinction is a tad terminologically misleading, since the current field of language acquisition is partly a subfield of modern psychology. Nonetheless, I retain the terms to show the origin of the relevant differences and take it that sufficient context is given to differentiate the two.}. In losing this vital distinction, willingly or unwillingly, Lado leads the reader and the subsequent researchers to confusingly collate two different definitions of transfer: that of \cite{mcgeoch_psychology_1942} and roughly that of \citeyear{james_principles_1890-1}. While the latter could deal with subtle, structural properties and employ copying, the former could not. McGeoch’s definition is especially unsuitable for linguistics because ``linguistic'' transfer (in acquiring gender, for example) is not a vague \textbf{influence on performance} -- recollect ``influence'' in McGeoch's definition --, it is a much more subtle structural \textbf{function}\footnote{I use function here in a pre-theoretic sense, roughly as mapping from one set to another. More on this, viz. one example of possible formalization and a more precise, yet still pre-theoretic description will be given in a later chapter (on algorithmic efficiency).} \textbf{within competence}\footnote{I am not aiming to define competence vs. performance distinction here, but a relevant introduction is given at the beginning of \cite{chomsky_aspects_1965}, some discussion relevant to SLA is given in \cite{epstein_universal_1996}.}. In other words, Lado collates under the same definition of transfer: 

\pex Lado's collation 
\a the cases where learners transfer lexicon settings (e.g., gender or some \(\theta\)-settings) or syntactic parameters (e.g., headedness setting or constraints on any given kind of movement) from a previous language\footnote{I further contend that whatever ``transfer'' is, the same process cannot be applied toward both idiosyncratic properties of language (viz. lexicon) and its syntax. This difference could not have appeared in Lado because Chomsky's work was not yet there, but it might be instructive to note this difference as well.}, with 
\a cases where learners memorize the verse faster because of previous training.
\xe 

While, trivially, these activities share some similarity or perhaps directionality at some level of abstraction, it is impossible to imagine them nesting under the same definition (unless the definition is indefensibly vague). No experimental design could possibly aim to investigate both such ``transfers''.  

We shall not attempt to formalize the Lado’s definition, since, as we have shown, it is a kluge of two, and our trying to put it in some structured form will only lead to further confusion. We will only prematurely mention that perhaps it is this Lado’s lack of terminological dexterity that will lead transfer to become a notion which, in \cite{wenk_interference_1974} and \cite{kellerman_towards_1977}’s words, can mean anything to anyone.         

\section{Tumultuous development of transfer in linguistics: 1957-1991}

In this section, we will take a brief look at some of the views, arguments, and questions which emerged within some very heated discussions of transfer over the course of 35 years after \cite{lado_linguistics_1957}’s work. Keeping it real, however, the discussion will not, and effectively cannot be posed to be comprehensive, since over those 35 years, many approaches to linguistics and \textit{ergo} to language acquisition emerged. The countless papers, conferences, and talks are impossible to exhaustively summarize or list even if one were to write a monograph on the subject. We, however, do not aim at such an exposition -- a holistic bird’s eye view is enough to grasp the terminological trends which we aim to review. The reason we end the section in 1991 is that the 1990s is the time when the transfer in modern meaning emerged in roughly generative linguistics. This is evidentiated by the sequences of papers by the same authors whose definitions of transfer, while different, did not evolve or evolved insignificantly: Schwartz and Sprouse (\citeyear{schwartz_l2_1996}) and Schwartz and Sprouse (\citeyear{schwartz_full_2021}); or Epstein \textit{et al}. (\citeyear{epstein_second_1996}), Flynn \textit{et al.} (\citeyear{flynn_cumulative-enhancement_2004}), Flynn (\citeyear{flynn_microvariation_2021}), and Fernández-Berkes and Flynn (\citeyear{fernandez-berkes_vindicating_2021}).


Back to 1960, we see how Lado’s work and diffusion of the definitions spread to other works. For example, Stephens writes: ``Transfer is more likely to take place when the thing to be transferred is \textbf{a generalization}, a conscious insight, a constant error to be dealt with, or a rule that can be understood'' \citep[][p. 1542, emphasis added]{stephens_transfer_1960}). Is transfer a generalization now? A conscious insight that came perhaps to be known as metalinguistic competence -- awareness of language structure which is akin to awareness that $proj_L(\Vec{x})=\Vec{x}^{\parallel}$?\footnote{Formula for orthogonal projection of a vector \citep[from][p. 61]{bretscher_linear_2018}} This is the first time around that generalization gets ``transferred''. This view (transfer operating over a generalization) contrasts heavily with later work. For example, Libuše Dušková dichotomized transfer and (over)generalization as a means of acquisition, pointing out that there isn’t only one way to acquire language, i.e. it isn’t all about transfer. She uses her experiment on acquiring English by Czech student who did not mark plurality in English, albeit Czech marks plurality \citep{duskova_sources_1969}; see also the discussion of this paper by Dušková, including findings contra the CA paradigm  \citep[][pp.. 14-17]{flynn_parameter-setting_1987}. 

Somewhat more principled accounts of transfer developed in the mid-1960s. For example, the notion of hierarchy of learning difficulty was introduced. Bowen and Martin write that ``assignment of an item [in a hierarchy of learning difficulty] is based on the premise that [positive] transfer from one language to another [...] becomes more difficult as the correspondences weaken'' \citep[][p. 292]{stockwell_grammatical_1965}. This aligned fine with the CA paradigm, but also somewhat refined the boundaries of transfer, being a yet another primordial version of later models of acquisition based on typological relationships between languages. 

Around this time, transfer took the central position in second language studies\footnote{Transfer even spread beyond language studies to, for example, poetics  \citep[see][]{abraham_towards_1973}.}, and critical views on transfer abounded. For example, Politzer, reflecting on transfer, writes that while that ``on a beaucoup étudié la question du transfert des connaissances d’une langue étrangère à une deuxième, sans parvenir à des conclusions définitives''\footnote{The question of transfer of knowledge from one language to another has been extensively studied, yet no definitive conclusions have been reached (translation is mine -- DMO.).}” \citep[][p. 1]{politzer_reflections_1965}. By the end of 1960s, Jakobovitz claims that while ``the literature on transfer (when the term is considered in its broadest sense) is possibly more extensive than that on any other topic in psychology and education [...] careful reviews of the vast literature pertaining to transfer are invariably pessimistic'' \citep[][p. 57]{jakobovits_second_1969}. The invariability which Jakobowitz alleges was perhaps too radical of a claim. While the pessimism was noticeable, not the least in the works of Jakobowitz himself, evidently there wasn’t enough of it to refine the terminology or give up the transfer altogether. While no definition was present in the paper, Jakobowits avers that ``similarities between two languages in terms of their surface features are more relevant to the operation of transfer effects than deep structure relations'' \citep[][p. 55]{jakobovits_second_1969}. It follows from this claim that ``surface features'' exist perhaps someplace different and are not connected to ``deep structure relations'' for how is it possible for surface structures to be considered on a separate basis and be thus ``more relevant''. This claim is at best putative and inconsistent with modern views on language architecture.  

Jacobowitz (\citeyear{jakobovits_second_1969}) was also the first one to offer the formalization of transfer. He writes that ``a general formulation of the transfer problem must deal with five basic elements: task A, task B, training or practice on task A, training or practice on task B, and the relation between task A and task B'' (59). Hence, for him, the transfer effect can primordially be expressed as

\pex
$P_{L_2}=f(P_{L_1},t_{L_2}, R_{L_1-L_2})$, where \\
$P_{L_1}$ is proficiency on Task A, \\
$P_{L_2}$ is proficiency on Task B, \\
$t_{L_2}$ is training in L2, \\
$R_{L_1-L_2}$ is some ``relation between $L_1$ and $L_2$''.\\ 
\xe

He goes through some modifications of this formula, separating transfer and deducing the formula which he sees as the definition of transfer. The detailed argumentation can be found in Jakobovits (\citeyear{jakobovits_second_1969}. pp. 59ff), but we will only consider some of the problems with his approach. The problem with this formalization is that, while attempting to deal with language (the paper was on ``second language learning''), it still operates within this kluge of definitions: we see Jakobowitz using the terms like ``proficiency on Task A'' which reminds us of Rugg (\citeyear{rugg_experimental_1916}) and McGeoch (\citeyear{mcgeoch_psychology_1942}). Once again, proficiency is a measure of performance (including that on task A) which has little to do with competence. While we no doubt, cannot get directly at competence (hence acceptability judgements, elicited imitation etc.), we do not describe or see acquisition in terms of performance. The (relevant) transfer (if any) being a part of establishing or amending individual language structure/architecture, no doubt, takes place at the level of competence\footnote{This logic holds still even if we stipulate surface structure vs. deep structure distinction of Jacobowitz.}. While discussing Lado, we decided against formalizing his definition seeing all of the pitfalls -- but Jakobowitz went down this road, and the obtained formalization has little value. What is ``training''? How can it be quantified? What is the function denoted as $f$? It appears as though ``formalization'' is no more than a fancy bit notation, void of clarity and precision.

The very first intentional distinction between two definitions Lado fused comes from Selinker: 

\begin{quote}
    I consider the following to be processes central to second-language learning: first, language transfer; second, transfer-of-training; third, strategies of second-language learning; fourth, strategies of second-language communication; and fifth, overgeneralization of TL [target language -- DMO] linguistic material \citep[][p. 216]{selinker_interlanguage_1972}.
\end{quote}

We are most interested in the seeming difference between the first two: language  transfer and transfer-of-training. Selinker explains:

\begin{quote}
    If it can be experimentally demonstrated that fossilizable items, rules, and subsystems which occur in IL [interlanguage --DMO] performance are a result of the NL [native language -- DMO], then we are dealing with the process of \textbf{language transfer}; if these fossilizable items, rules, and subsystems are a result of identifiable items in training procedures, then we are dealing with the process known as the \textbf{transfer-of-training}... \citep[][p. 216, emphasis added]{selinker_interlanguage_1972}
\end{quote}

This is an attempt to deal with the problem of collation of definitions mentioned above. Even attempting to deal with it is great. Yet, there are a number of questions in relation to the distinction which Selinker draws. First of all, what is fossilization and what are the ``items, rules, and subsystems'' which are ``fossilizable''\footnote{How is fossilizable vs. non-fossilizaable distinction drawn? This is painfully reminiscent of Lado's ``transferability''.}? If we turn to the standard definition, only errors are referred to as being fossilized in modern linguodidactics (not linguistics, though), and fossilized errors are ``errors which a learner does not stop making and which last for a long time, even for ever [...]. Fossilization of error often happens when learners [… ] have no communicative reason to improve their language'' \citep[][p. 63]{spratt_tkt_2011}. Fossilization hence appears to be of purely applied nature, i.e. having little to do with primarily unconscious acquisition of language system(s), whatever those systems are\footnote{I remain optimistic about fossilization, in language teaching terms, being just a lag or a bug in externalization, accountable for non-nativelike speech. It is my hope that ``fossilizzation'' has nothing do to with competence, only with performance: the reason being that if it has to with competence, the learners would have problems parsing the grammatically convention input with the ``fossilized'' parts of their competence. Receptive bilinguals seem to be a case in point to argue that performance has little to do with competence (cf. \citep[][\textit{inter alia}]{sherkina-lieber_grammar_2011, sherkina-lieber_classification_2020}}. To support that, we see that Selinker operates within performance which we have already taken to be an unreliable narrator of language acquisition.

The distinction which Selinker draws does not appear to be viable. His transfer-of-training is alleged to stem from ``identifiable items in training procedures''. However, what are these items? If a learner whose $L_1$ does not mark plural morphologically overgeneralizes third person singular inflection of English and subsequently uses it for non-conforming forms like oxen and sheep (cf. *oxs or *oxes, *sheeps), is that transfer-of-training? Similar generalization and abstraction processes, no doubt, occur on a much deeper level \citep[cf.][pp. 68, 122, 170 fn30, and section 11.2 generally]{lust_child_2006} and are far from easily identifiable transfer-of-(conscious)-training. Therefore, drawing the distinction by dichotomizing transfer as coming either from $L_1$ or from training fails to account for the elaborate process of language acquisition.

One positive feature of the definition above is that there are clear elements in it (the four in the original quote). Later works overlooked this clarity and fused the four elements all over again. As such, Shachter writes that ``if the constructions are similar in the learner’s mind, [they] will transfer his \textit{native language strategy} to the target language'' \citep[][p. 212, emphasis added]{schachter_error_1974}. What the strategy is was left undefined. The objects of transfer varied vastly: while Shachter’s transfer was that of strategies, Taylor’s was again of ``structures'' \citep[][p. 75]{taylor_use_1975}, and so forth. 

In subsequent years, attempts to distinguish transfer from interference ensued. One of those was Kellerman’s paper on ``strategy of transfer''\footnote{We shall not consider these proposals of Kellerman in further detail because the terms ``interenfence'' and ``cross-linguistic influence'' deserve a separate investigation, even though a much more narrow one than that of transfer.}:

\begin{quote}
    the connection between transfer experiments  in the laboratory (which is the place where the term ``interference'' strictly belongs) and transfer in second language learning have been shown to be very tenuous, with many writers being reluctant to link the two in any significant fashion \citep[][p. 61]{kellerman_towards_1977}
\end{quote}

Similar logic appeared in Kellerman (\citeyear{kellerman_transfer_1979}). This could be seen as an attempt to separate language learning or language teaching from language acquisition: while ``fossilization'', ``strategies,'' and other CA heritage are relevant mostly to the classroom practice (which was, naturally, the principal motivation of Lado’s CA paradigm), Kellerman attempts to draw a distinction between acquisition inquiry and language classroom. Judging by the work which followed, he did not succeed in his quest: the studies that followed him did not adopt his distinction. Moreover, ``transfer of communication strategies'' appeared \citep[cf.][he also called it ``reflexation'']{zobl_developmental_1980}). In other words, everything was still claimed to have transferred from $L_1$ to $L_2$, but nobody really knew what transfer was\footnote{The chief problem, in my eyes, is that nobody tried to define $L_1$ and $L_2$.} (cf. lack of definitions in contemporary papers \citep[e.g.,][]{johnson_factors_1989}. Evidently, subdoxasticity persisted and prevailed. 

%\footnote{An instructive paper on why linguists generally do need language acquirers (and hopefully acquisitionists as well) was written by Comrie (\citeyear{comrie_why_1984}) in tha volume. While it focuses mostly on the Keenan-Comrie Accessibility Hierarchy and the related work in SLA, it is the rhetoric and the spirit of the paper which holds instructive value, calling on cooperation and collaboration (while paying close attention to each other’s work).}

One of the first edited volumes on transfer appeared in the early 1980s: Gass and Salinker (\citeyear{gass_language_1983}). The papers which concerned linguistic transfer in the volume were those of Olshtain, Adjemian, Broselow, Scarcella, Bartelt, and Jordens. Additionally, the first studies on attrition appeared and also heavily relied on transfer theory in their theorizing \citep[e.g.,][]{olshtain_is_1989}, on typological factors of the CA kind and dominant language; \citep{berman_features_1983, sharwood_smith_first_1983}. No further definitions appeared; again, transfer was probably taken to be too trivial -- or, at this point -- at the same time far too complex to define. Those interested in papers on transfer within the 1990s (apart from those referenced above), are directed to Koda (\citeyear{koda_use_1990}; transfer in processing) and Jaworski (\citeyear{jaworski_acquisition_1990}; transfer vs transfer-of-training).

The last paper we will focus on is ``Now you see it, now you don’t'' by Kellerman (\citeyear{kellerman_now_1983}). He writes: 

\begin{quote}
    While it goes without saying that language learners have boundless opportunities to develop typological knowledge and that this knowledge will affect the nature of their output, to the extent that the learner's overall approach to the $L_2$ may be one of recreation or restructuring \citep{corder_language_1979}, it is also clear that the possession of such knowledge is generally orientational and is not in itself sufficient to to account for specific linguistic behavior. That is to say, \textbf{typological similarity will not prove to be an adequately principled basis for the prediction of cross-linguistic effects}. (1983:116, emphasis added)
\end{quote}

This part, while not mentioning transfer directly, attacks the CA paradigm, its conceptual interpretation of transfer, and, indirectly, the vaguely similar modern models which rely on typology or (psycho)typology, and are rooted in transfer. This Kellerman’s passage offers a segway to the modern discussions of transfer. As we have seen from the discussion above, exhaustively all linguistic discussion, spurred by Lado, focused on typology and superficial differences between languages. It is not imperative, however, that one proceeds with this dubious relic.

Entirely consonant to this view are Fernández-Berkes and Flynn: “...an assumption that reduces the $S_0$ of the multilingual learner when facing new input of a specific language to  the result of one intervening variable [typology -- DMO] \textbf{does not} provide an explanatorily adequate language acquisition theory” \citep[][p. 31, emphasis added]{fernandez-berkes_vindicating_2021}. It is on this note of consonance\footnote{We consciously overlook the fact that Kellerman is talking about typology and cross-linguistic influence, and Fernández-Berkes and Flynn are talking about typology and theory of language acquisition; the rhetoric and the target of their message are clearly closely similar if not identical.} between the papers almost forty years apart that we conclude the main chronological state-of-art analysis.

\section{Transfer: looking forward having looked back} \label{transf-forw-back}

If one were to go back and look at the problems we outlined apropos Lado’s definition, transfer-of-training and linguistic transfer could easily transpire to be related to E-language and I-language respectively, and Jakobowitz was far from \cite{chomsky_aspects_1965}'s sense of those terms. It appears to be quite trivial that we ought not base psycholinguistic inquiry on E-language or transfer-of-training, and while most of current studies are not actually interested in investigating E-language \textit{per se}, the ambiguous terminology, and by extension the confusion, persists.

However, there is one problem which is much graver: if transfer in its roots stemmed from psychology, ought we not reconsider its value and meaning for generative linguistic inquiry? I believe the answer is yes. Adopting the term itself is not too problematic, it's the ever-haunting heritage of ill-definedness and behaviorism that is problematic. The behavioristic scaffolding of transfer seems to have hindered the clarity and transparency of generative studies too. Take, for example, \cite{ozernyi_l1_2021}. In that study, Ozernyi does not clarify what transfer is, but talks of transfer. Moreover, the two ``properties'' Ozernyi considers are highly complex: definiteness and gender. Gender might as well turn out to be a bundle of properties, depending on the definition of ``property'' and theory of cross-linguistic variation (e.g., parameters) one adopts. Ozernyi correctly draws distinctions between definitenes, determinacy, and specificity -- but does not even mention what the alleged ``transfer'' - as an operation -- applies to, how it proceeds, and what its output is. This, of course, heavily hinders meaningful interpretation of the results of the study. While in all likelihood the overall conclusion that ``contrary to the models which suggested that one language plays a predominant role in subsequent language acquisition, all languages influence $L_4$`` \citep[][p. 21]{ozernyi_l1_2021} is likely to be correct in the long run, the theoretical scaffolding of the paper is at best questionable.

As such, the situation with ``transfer'' is rather dire, really. In what follows I will very briefly sketch out a proposal to abolish transfer in favor of some other terminological notion. What I stipulate is that there is no need for transfer \textit{per se}. Transfer carries with it a certain connotation of relinquishing or locomotion, which might have possibly caused the researchers to talk about ``copying'' relationships or different ``mental representations'' for each of n-ary language acquisition. A helpful visual aid is in \cite{westergaard_microvariation_2021}: it seems as if on the diagrams in her paper each $L_1$, $L_2$, and $L_3$ have separate ``vials'' where the representations are located\footnote{Those ``vials'' can be effectively reimagined as sets, but the questions persists: sets of what? why are they separate?.}. Indeed, such an image of distinct and separate mental representations is ubiquitous: \citet[pp. 150, 158]{kroll_category_1994}; \citet{grosjean_bilinguals_2001}; \citet[pp. 106ff and adapted visuals references therein]{kroll_cognitive_2003}; \citet[pp. 146f]{pavlenko_conceptual_2009}; \citet[p. 752]{riehl_mental_2010}; \citet[p. 268]{benati_input_2017}; and so forth. It is this separation and ``transfer'' from one ``vial'' to another one that I dispute passionately. \cite{sharwood_smith_language_2021} aptly noted that ``[the] ‘move-from-one-location-to-another’ notion is misleading and unnecessary'' (413). 

I see no reason behind assuming such a distinction between different (groups of) mental representations. What is it based on? It is entirely possible that the mental representations function as sets, with the corresponding nonempty intersections between those sets\footnote{A particular case of this set-theoretic conception is when there is no intersection between sets of $L_1$ and $L_2$ (and $L_3...$) -- but I find that this case is repealed through mere existence of language universals. A stronger case comes from recent Chomsky's work which assumes that cross-linguistic variation is limited to externalization. Then, all we have is one set. This a very strong claim with very serious implications for acquisition prompting us to rethink what it is we are trying to investigate. This assumption will not be tacked here.}. In such a continuum, the language of an individual is a set of mental representations of properties\footnote{Depending on the parametric framework one chooses, this could be e.g., parameters, etc.} relating to one or more languages. For example, V2 parameter for an individual who speaks Russian, Ukrainian, and English, and is acquiring German does not transfer from one language to another one, nor is it distinct for each language. Instead, it is one parameter which lies at the intersection of all four subsets of mental representations. Similarly, the SOV parameter for subordinate clauses is only within the German subset, and overt V-T movement is at the intersection of German and English. The parametric theory or analogous alternatives are, doubtlessly, central to such a framework.

Assuming that there is a copy of a given parameter, feature, or property for each language is just redundant and alien to the minimalist framework (\citep[cf.][]{chomsky_phases_2005, berwick_biolinguistic_2011}. Additionally, this approach resolves the innumerable problems faced by the models based on typology. Namely, for an acquirer of Chinese who speaks Polish and Belarusian, which one is closer to Chinese to be transferred at the ``initial stages''\footnote{The notion of ``initial state'' and ``initial stages'' is highly questionable on another account and will be scrutinized in the next chapter.}? Otherwise unresolvable, this question does not appear in the set-like framework we described. Additionally, this view finally does away with inaccessibility of UG later in the adult acquisition  \citep[see][]{lenneberg_biological_1967, clahsen_availability_1986, schachter_what_1989} as it presupposes accessibility to UG as the only means of successful acquisition\footnote{Again, more on this in later chapters, for now let us leave it as a stipulation.}. The models of the cited authors would therefore fall within the scope of transfer-of-training, being out of scope of generative inquiry.  

Yet another problem with CA-based transfer or typologically-based copying\footnote{And, by extension, the models which employ these or hghly similar concepts like TPM of \cite{rothman_linguistic_2015}, or FTFA of \cite{schwartz_l2_1996}.} is that the learners never seem to make some mistakes which would be expected for full transfer. One such example would be using the declension paradigm for nouns or adjectives from their mother language with the English stems. Not having acquired the analytical geninive of English (‘of’), this should be the case. However, Czech learners of English never seem to use Czech endings while assigning a case for English stems of either adjectives or nouns. The constructions like \textit{*the book studentů} (cf. the respective Czech \textit{kniha studentů} and English \textit{the book of students}) are ruled out even for the \textit{ab initio} learners, even for those lacking relevant lexicon. Why is this the case? Dušková’s study cited above is yet another case in point, and the further examples are endless. Wholesale transfer and ``typology'' do not offer a descriptively adequate account of acquisition. I thus support the view of language representations as a continuum, not as a disrupted or eclectic collection of parameters and principles. One of the remaining questions which we did not tackle in this paper and which remained tacit throughout all the studies is that of criteria for transfer -- what is deemed similar enough to be transferred or merged? This question remains unresolved, but I believe that, minimalistic in its spirit, the theory of features from syntax can be abstractly extended to language acquisition. Another approach are micro-cues \citep{westergaard_acquisition_2008, westergaard_linguistic_2014} or triggers \citep[e.g.,][and the work that followed]{gibson_triggers_1994}.

Lastly, before moving to conclusion, allow us one brief exploratory digression. Over the course of the paper, it effectively appears that acquisitionists are sloppy with the terminology they use, perhaps even unwilling to accept new terminology. We then became interested whether this is the case in other linguistic subfields (e.g. generative syntax). It did not require much effort, and we arbitrarily picked a notion in syntax (“sluicing”) and two years (2009 and 2010) to check whether syntacticians define the basic terminology they use. Out of 10 papers found by Scopus search for “sluicing” which were on linguistic sluicing (as opposed to the hydrological one), eight included the definition of reasonable clarity \citep{park_resolving_2009, van_craenenbroeck_syntax_2009, khan_resolving_2010, kimura_wh--situ_2010, saab_silent_2010, van_craenenbroeck_invisible_2010, yoshida_antecedent-contained_2010, poirier_real-time_2010}, and two did not include the definition (\citet{hall_subsentential_2009}, but sluicing was very peripheral to her paper; \citet{arregi_ellipsis_2010}, but they included examples). That is to say that perhaps acquisitionists should be as dextrous as syntacticians are, particularly since we deal with arguably more ephemeral and more abstruse concepts.

\section{Conclusive remarks}

At this stage of development of linguistic inquiry, it is essential to abjure the blind adherence to atavistic transfer. It is incumbent upon any theory of $L_n$ acquisition ``to establish how the [$L_n$] learner determines reasonableness, systematicity, explicitness, [and -- DMO] logicalness'' \citep[][p. 28]{flynn_parameter-setting_1987}. In order to face such a feat, the theory itself should be ``reasonable'', and -- crucially -- formally sufficient. It is high time for the generative language acquisitionists, famed for our allegiance to structure, clarity, and logic, to alter the outworn parts of our terminological base and, moving to the new levels of sophistication, do justice to the pursuit of idealized and explanatory adequate acqusition process. This, among all other reasons, is why it is vitally important to contemplate and reflect on the rise and fall of linguistic transfer.

\bibliographystyle{acl_natbib}
\bibliography{references}
\end{document}
